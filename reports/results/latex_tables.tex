% LaTeX Tables untuk Paper - Clustering Diabetes
% Dihasilkan secara otomatis dari Notebook 04b

% ============================================================
% TABEL 1: PERBANDINGAN PERFORMA METODE CLUSTERING
% ============================================================

\begin{table}[htbp]
\centering
\caption{Perbandingan Performa Metode Clustering pada Dataset Diabetes}
\label{tab:clustering_performance}
\begin{tabular}{llcccccc}
\toprule
\textbf{Metode} & \textbf{Fitur} & \textbf{k} & \textbf{Silhouette} & \textbf{Davies-Bouldin} & \textbf{Calinski-H.} & \textbf{Peningkatan} \\
\midrule
K-Means (Baseline) & 69 fitur asli & 2 & 0.223 & 2.330 & 5026 & — \\
K-Means (Latent, k=2) & 16 fitur laten & 2 & 0.437 & 1.257 & 18070 & +96.0\% \\
\textbf{K-Means (Latent, k=3)} & 16 fitur laten & 3 & \textbf{0.470} & \textbf{1.029} & \textbf{20321} & \textbf{+110.8\%} \\
K-Means (Latent, k=4) & 16 fitur laten & 4 & 0.440 & 0.897 & 17869 & +97.3\% \\
K-Means (Latent, k=5) & 16 fitur laten & 5 & 0.458 & 0.852 & 17484 & +105.4\% \\
DEC (k=3) & 16 fitur laten (fine-tuned) & 3 & 0.234 & 2.309 & 1704 & +4.8\% \\
\bottomrule
\end{tabular}
\begin{tablenotes}
\small
\item Catatan: Nilai Silhouette yang lebih tinggi dan Davies-Bouldin yang lebih rendah menunjukkan kualitas clustering yang lebih baik. K-Means (Latent, k=3) merupakan metode terbaik dengan peningkatan 110.8\% dibanding baseline.
\end{tablenotes}
\end{table}


% ============================================================
% TABEL 2: KARAKTERISTIK CLUSTER
% ============================================================

\begin{table}[htbp]
\centering
\caption{Karakteristik Tiga Cluster Pasien Diabetes (K-Means Latent, k=3)}
\label{tab:cluster_characteristics}
\begin{tabular}{clrrr}
\toprule
\textbf{Cluster} & \textbf{Label Klinis} & \textbf{Ukuran} & \textbf{Persentase} & \textbf{Tingkat Readmisi} \\
\midrule
C0 & Standard Care (Lower Complexity) & 70,631 & 81.7\% & 0.46\% \\
C1 & Medication-Adjusted Care & 10,583 & 12.2\% & 0.47\% \\
C2 & Extended Hospital Stay & 5,287 & 6.1\% & 0.46\% \\
\bottomrule
\end{tabular}
\begin{tablenotes}
\small
\item Catatan: Uji Chi-square menunjukkan tidak ada perbedaan signifikan dalam tingkat readmisi antar cluster ($p=0.064$). Cluster merepresentasikan fenotipe kompleksitas perawatan, bukan stratifikasi risiko readmisi.
\end{tablenotes}
\end{table}


% ============================================================
% TABEL 3: FITUR PEMBEDA UTAMA PER CLUSTER
% ============================================================

\begin{table}[htbp]
\centering
\caption{Lima Fitur Pembeda Utama untuk Setiap Cluster}
\label{tab:top_features}
\small
\begin{tabular}{clccp{5cm}}
\toprule
\textbf{Cluster} & \textbf{Fitur} & \textbf{Perbedaan} & \textbf{Cohen's d} & \textbf{Interpretasi} \\
\midrule
\multicolumn{5}{l}{\textbf{C0: Standard Care (Lower Complexity)}} \\
 & lab\_intensity\_medium & -0.336 & -0.307 & ↓ Lebih rendah \\
 & citoglipton & -0.291 & -0.277 & ↓ Lebih rendah \\
 & los\_long & -0.245 & -0.219 & ↓ Lebih rendah \\
 & medication\_down\_count & -0.166 & -0.166 & ↓ Lebih rendah \\
 & procedure\_lab\_ratio & -0.165 & -0.166 & ↓ Lebih rendah \\
\midrule
\multicolumn{5}{l}{\textbf{C1: Medication-Adjusted Care}} \\
 & lab\_intensity\_medium & +0.467 & 0.407 & ↑ Lebih tinggi \\
 & citoglipton & +0.371 & 0.344 & ↑ Lebih tinggi \\
 & clinical\_complexity\_score & -0.233 & -0.187 & ↓ Lebih rendah \\
 & medication\_down\_count & +0.227 & 0.228 & ↑ Lebih tinggi \\
 & procedure\_lab\_ratio & +0.188 & 0.190 & ↑ Lebih tinggi \\
\midrule
\multicolumn{5}{l}{\textbf{C2: Extended Hospital Stay}} \\
 & los\_long & +0.714 & 0.517 & ↑ Lebih tinggi \\
 & los\_medium & -0.440 & -0.387 & ↓ Lebih rendah \\
 & resource\_utilization\_score & -0.319 & -0.300 & ↓ Lebih rendah \\
 & frequent\_visitor & -0.173 & -0.176 & ↓ Lebih rendah \\
 & utilization\_high & -0.125 & -0.157 & ↓ Lebih rendah \\
\bottomrule
\end{tabular}
\begin{tablenotes}
\small
\item Catatan: Perbedaan menunjukkan rata-rata terstandarisasi cluster dibanding cluster lainnya. Cohen's d mengindikasikan ukuran efek (|d| $\geq$ 0.2: kecil, $\geq$ 0.5: sedang, $\geq$ 0.8: besar). Semua perbedaan signifikan secara statistik ($p < 0.001$).
\end{tablenotes}
\end{table}
